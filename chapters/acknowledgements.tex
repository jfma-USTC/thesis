% !TeX root = ../main.tex

\begin{acknowledgements}
	
	本科四年的学习即将落幕,回首望去,一路上有许多帮助我成长的老师和同学,
	心中的感激之情难以言表。
	
	首先感谢从大一新生研讨课开始就一直指导我的杜俊老师,
	感谢杜老师让我明白了科研的价值和意义,
	让我看到了每一个学术成果的背后,都需要研究者对自己领域有着满腔的热情和一颗不断探索的心。
	每次和杜老师的谈话我总能受益良多,感谢老师在我迷茫时开导我,在我退缩时鼓励我,
	在我自满时点醒我。凡此种种,不胜枚举。
	在这次毕业设计中,老师在论文选题、实验设计等方面都给予了我很多的帮助与启发,
	再次对老师的悉心指导表示诚挚的谢意。
	
	感谢汪子锐师兄、王嘉明师兄在我初入科研之路时对我的耐心教导与帮助。
	汪子锐师兄给予了我很多关心和鼓励,从实验细节到科研规划,师兄都事无巨细的帮我排难解惑,
	本次论文的诸多细节也受益于师兄的指导才得以成型。
	王嘉明师兄在我研究受阻时给予了我很大的帮助,
	师兄的学习笔记让我在研究Kaldi这一复杂的开源代码库时受益匪浅。
	
	感谢实验室的师兄师姐们给予我的鼓励和帮助,
	感谢在科研之路上互相勉励的伙伴们给予我的关心和支持。
	感谢远在浙江的父亲母亲辛苦地将我养育成人,
	父母给予我的精神财富使我受用终生。
	感谢姐姐的在疫情期间的种种帮助,
	姐姐的敦促与教诲是我不断努力的一大动力。
	同时感谢潘秀秀同学在生活上的支持与陪伴。
	正是你们的付出,让我在四年的学习生涯里多了欢声笑语,少了忧愁烦闷。
	
	科大是一所特别的校园,每位学子都可以在这里找到真实的自己。
	岁月流金,希望包括我在内的经历过科大生活的人,
	在以后的人生里都能从这段岁月中获得长久的力量,
	攀顶一座又一座的高峰。
	
\end{acknowledgements}
